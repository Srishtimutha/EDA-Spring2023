% Options for packages loaded elsewhere
\PassOptionsToPackage{unicode}{hyperref}
\PassOptionsToPackage{hyphens}{url}
%
\documentclass[
]{article}
\usepackage{amsmath,amssymb}
\usepackage{lmodern}
\usepackage{iftex}
\ifPDFTeX
  \usepackage[T1]{fontenc}
  \usepackage[utf8]{inputenc}
  \usepackage{textcomp} % provide euro and other symbols
\else % if luatex or xetex
  \usepackage{unicode-math}
  \defaultfontfeatures{Scale=MatchLowercase}
  \defaultfontfeatures[\rmfamily]{Ligatures=TeX,Scale=1}
\fi
% Use upquote if available, for straight quotes in verbatim environments
\IfFileExists{upquote.sty}{\usepackage{upquote}}{}
\IfFileExists{microtype.sty}{% use microtype if available
  \usepackage[]{microtype}
  \UseMicrotypeSet[protrusion]{basicmath} % disable protrusion for tt fonts
}{}
\makeatletter
\@ifundefined{KOMAClassName}{% if non-KOMA class
  \IfFileExists{parskip.sty}{%
    \usepackage{parskip}
  }{% else
    \setlength{\parindent}{0pt}
    \setlength{\parskip}{6pt plus 2pt minus 1pt}}
}{% if KOMA class
  \KOMAoptions{parskip=half}}
\makeatother
\usepackage{xcolor}
\usepackage[margin=2.54cm]{geometry}
\usepackage{color}
\usepackage{fancyvrb}
\newcommand{\VerbBar}{|}
\newcommand{\VERB}{\Verb[commandchars=\\\{\}]}
\DefineVerbatimEnvironment{Highlighting}{Verbatim}{commandchars=\\\{\}}
% Add ',fontsize=\small' for more characters per line
\usepackage{framed}
\definecolor{shadecolor}{RGB}{248,248,248}
\newenvironment{Shaded}{\begin{snugshade}}{\end{snugshade}}
\newcommand{\AlertTok}[1]{\textcolor[rgb]{0.94,0.16,0.16}{#1}}
\newcommand{\AnnotationTok}[1]{\textcolor[rgb]{0.56,0.35,0.01}{\textbf{\textit{#1}}}}
\newcommand{\AttributeTok}[1]{\textcolor[rgb]{0.77,0.63,0.00}{#1}}
\newcommand{\BaseNTok}[1]{\textcolor[rgb]{0.00,0.00,0.81}{#1}}
\newcommand{\BuiltInTok}[1]{#1}
\newcommand{\CharTok}[1]{\textcolor[rgb]{0.31,0.60,0.02}{#1}}
\newcommand{\CommentTok}[1]{\textcolor[rgb]{0.56,0.35,0.01}{\textit{#1}}}
\newcommand{\CommentVarTok}[1]{\textcolor[rgb]{0.56,0.35,0.01}{\textbf{\textit{#1}}}}
\newcommand{\ConstantTok}[1]{\textcolor[rgb]{0.00,0.00,0.00}{#1}}
\newcommand{\ControlFlowTok}[1]{\textcolor[rgb]{0.13,0.29,0.53}{\textbf{#1}}}
\newcommand{\DataTypeTok}[1]{\textcolor[rgb]{0.13,0.29,0.53}{#1}}
\newcommand{\DecValTok}[1]{\textcolor[rgb]{0.00,0.00,0.81}{#1}}
\newcommand{\DocumentationTok}[1]{\textcolor[rgb]{0.56,0.35,0.01}{\textbf{\textit{#1}}}}
\newcommand{\ErrorTok}[1]{\textcolor[rgb]{0.64,0.00,0.00}{\textbf{#1}}}
\newcommand{\ExtensionTok}[1]{#1}
\newcommand{\FloatTok}[1]{\textcolor[rgb]{0.00,0.00,0.81}{#1}}
\newcommand{\FunctionTok}[1]{\textcolor[rgb]{0.00,0.00,0.00}{#1}}
\newcommand{\ImportTok}[1]{#1}
\newcommand{\InformationTok}[1]{\textcolor[rgb]{0.56,0.35,0.01}{\textbf{\textit{#1}}}}
\newcommand{\KeywordTok}[1]{\textcolor[rgb]{0.13,0.29,0.53}{\textbf{#1}}}
\newcommand{\NormalTok}[1]{#1}
\newcommand{\OperatorTok}[1]{\textcolor[rgb]{0.81,0.36,0.00}{\textbf{#1}}}
\newcommand{\OtherTok}[1]{\textcolor[rgb]{0.56,0.35,0.01}{#1}}
\newcommand{\PreprocessorTok}[1]{\textcolor[rgb]{0.56,0.35,0.01}{\textit{#1}}}
\newcommand{\RegionMarkerTok}[1]{#1}
\newcommand{\SpecialCharTok}[1]{\textcolor[rgb]{0.00,0.00,0.00}{#1}}
\newcommand{\SpecialStringTok}[1]{\textcolor[rgb]{0.31,0.60,0.02}{#1}}
\newcommand{\StringTok}[1]{\textcolor[rgb]{0.31,0.60,0.02}{#1}}
\newcommand{\VariableTok}[1]{\textcolor[rgb]{0.00,0.00,0.00}{#1}}
\newcommand{\VerbatimStringTok}[1]{\textcolor[rgb]{0.31,0.60,0.02}{#1}}
\newcommand{\WarningTok}[1]{\textcolor[rgb]{0.56,0.35,0.01}{\textbf{\textit{#1}}}}
\usepackage{graphicx}
\makeatletter
\def\maxwidth{\ifdim\Gin@nat@width>\linewidth\linewidth\else\Gin@nat@width\fi}
\def\maxheight{\ifdim\Gin@nat@height>\textheight\textheight\else\Gin@nat@height\fi}
\makeatother
% Scale images if necessary, so that they will not overflow the page
% margins by default, and it is still possible to overwrite the defaults
% using explicit options in \includegraphics[width, height, ...]{}
\setkeys{Gin}{width=\maxwidth,height=\maxheight,keepaspectratio}
% Set default figure placement to htbp
\makeatletter
\def\fps@figure{htbp}
\makeatother
\setlength{\emergencystretch}{3em} % prevent overfull lines
\providecommand{\tightlist}{%
  \setlength{\itemsep}{0pt}\setlength{\parskip}{0pt}}
\setcounter{secnumdepth}{-\maxdimen} % remove section numbering
\ifLuaTeX
  \usepackage{selnolig}  % disable illegal ligatures
\fi
\IfFileExists{bookmark.sty}{\usepackage{bookmark}}{\usepackage{hyperref}}
\IfFileExists{xurl.sty}{\usepackage{xurl}}{} % add URL line breaks if available
\urlstyle{same} % disable monospaced font for URLs
\hypersetup{
  pdftitle={Assignment 4: Data Wrangling},
  pdfauthor={Srishti Mutha},
  hidelinks,
  pdfcreator={LaTeX via pandoc}}

\title{Assignment 4: Data Wrangling}
\author{Srishti Mutha}
\date{Spring 2023}

\begin{document}
\maketitle

\hypertarget{overview}{%
\subsection{OVERVIEW}\label{overview}}

This exercise accompanies the lessons in Environmental Data Analytics on
Data Wrangling

\hypertarget{directions}{%
\subsection{Directions}\label{directions}}

\begin{enumerate}
\def\labelenumi{\arabic{enumi}.}
\tightlist
\item
  Rename this file
  \texttt{\textless{}FirstLast\textgreater{}\_A04\_DataWrangling.Rmd}
  (replacing \texttt{\textless{}FirstLast\textgreater{}} with your first
  and last name).
\item
  Change ``Student Name'' on line 3 (above) with your name.
\item
  Work through the steps, \textbf{creating code and output} that fulfill
  each instruction.
\item
  Be sure to \textbf{answer the questions} in this assignment document.
\item
  When you have completed the assignment, \textbf{Knit} the text and
  code into a single PDF file.
\end{enumerate}

The completed exercise is due on Friday, Feb 20th @ 5:00pm.

\hypertarget{set-up-your-session}{%
\subsection{Set up your session}\label{set-up-your-session}}

1a. Load the \texttt{tidyverse}, \texttt{lubridate}, and \texttt{here}
packages into your session.

1b. Check your working directory.

1c. Read in all four raw data files associated with the EPA Air dataset,
being sure to set string columns to be read in a factors. See the README
file for the EPA air datasets for more information (especially if you
have not worked with air quality data previously).

\begin{enumerate}
\def\labelenumi{\arabic{enumi}.}
\setcounter{enumi}{1}
\tightlist
\item
  Apply the \texttt{glimpse()} function to reveal the dimensions, column
  names, and structure of each dataset.
\end{enumerate}

\begin{Shaded}
\begin{Highlighting}[]
\CommentTok{\# 1a}
\FunctionTok{library}\NormalTok{(tidyverse)}
\FunctionTok{library}\NormalTok{(lubridate)}
\FunctionTok{library}\NormalTok{(here)}
\CommentTok{\# 1b}
\FunctionTok{getwd}\NormalTok{()}
\end{Highlighting}
\end{Shaded}

\begin{verbatim}
## [1] "/Users/cherry/Desktop/EDA/EDA-Spring2023"
\end{verbatim}

\begin{Shaded}
\begin{Highlighting}[]
\CommentTok{\# 1c}
\NormalTok{EPAPM19 }\OtherTok{\textless{}{-}} \FunctionTok{read.csv}\NormalTok{(}\StringTok{"./Data/Raw/EPAair\_PM25\_NC2019\_raw.csv"}\NormalTok{, }\AttributeTok{stringsAsFactors =} \ConstantTok{TRUE}\NormalTok{)}
\NormalTok{EPAPM18 }\OtherTok{\textless{}{-}} \FunctionTok{read.csv}\NormalTok{(}\StringTok{"./Data/Raw/EPAair\_PM25\_NC2018\_raw.csv"}\NormalTok{, }\AttributeTok{stringsAsFactors =} \ConstantTok{TRUE}\NormalTok{)}
\NormalTok{EPAO319 }\OtherTok{\textless{}{-}} \FunctionTok{read.csv}\NormalTok{(}\StringTok{"./Data/Raw/EPAair\_O3\_NC2019\_raw.csv"}\NormalTok{, }\AttributeTok{stringsAsFactors =} \ConstantTok{TRUE}\NormalTok{)}
\NormalTok{EPAO318 }\OtherTok{\textless{}{-}} \FunctionTok{read.csv}\NormalTok{(}\StringTok{"./Data/Raw/EPAair\_O3\_NC2018\_raw.csv"}\NormalTok{, , }\AttributeTok{stringsAsFactors =} \ConstantTok{TRUE}\NormalTok{)}

\CommentTok{\# 2}
\FunctionTok{glimpse}\NormalTok{(EPAPM19)}
\end{Highlighting}
\end{Shaded}

\begin{verbatim}
## Rows: 8,581
## Columns: 20
## $ Date                           <fct> 01/03/2019, 01/06/2019, 01/09/2019, 01/~
## $ Source                         <fct> AQS, AQS, AQS, AQS, AQS, AQS, AQS, AQS,~
## $ Site.ID                        <int> 370110002, 370110002, 370110002, 370110~
## $ POC                            <int> 1, 1, 1, 1, 1, 1, 1, 1, 1, 1, 1, 1, 1, ~
## $ Daily.Mean.PM2.5.Concentration <dbl> 1.6, 1.0, 1.3, 6.3, 2.6, 1.2, 1.5, 1.5,~
## $ UNITS                          <fct> ug/m3 LC, ug/m3 LC, ug/m3 LC, ug/m3 LC,~
## $ DAILY_AQI_VALUE                <int> 7, 4, 5, 26, 11, 5, 6, 6, 15, 7, 14, 20~
## $ Site.Name                      <fct> Linville Falls, Linville Falls, Linvill~
## $ DAILY_OBS_COUNT                <int> 1, 1, 1, 1, 1, 1, 1, 1, 1, 1, 1, 1, 1, ~
## $ PERCENT_COMPLETE               <dbl> 100, 100, 100, 100, 100, 100, 100, 100,~
## $ AQS_PARAMETER_CODE             <int> 88502, 88502, 88502, 88502, 88502, 8850~
## $ AQS_PARAMETER_DESC             <fct> Acceptable PM2.5 AQI & Speciation Mass,~
## $ CBSA_CODE                      <int> NA, NA, NA, NA, NA, NA, NA, NA, NA, NA,~
## $ CBSA_NAME                      <fct> "", "", "", "", "", "", "", "", "", "",~
## $ STATE_CODE                     <int> 37, 37, 37, 37, 37, 37, 37, 37, 37, 37,~
## $ STATE                          <fct> North Carolina, North Carolina, North C~
## $ COUNTY_CODE                    <int> 11, 11, 11, 11, 11, 11, 11, 11, 11, 11,~
## $ COUNTY                         <fct> Avery, Avery, Avery, Avery, Avery, Aver~
## $ SITE_LATITUDE                  <dbl> 35.97235, 35.97235, 35.97235, 35.97235,~
## $ SITE_LONGITUDE                 <dbl> -81.93307, -81.93307, -81.93307, -81.93~
\end{verbatim}

\begin{Shaded}
\begin{Highlighting}[]
\FunctionTok{glimpse}\NormalTok{(EPAPM18)}
\end{Highlighting}
\end{Shaded}

\begin{verbatim}
## Rows: 8,983
## Columns: 20
## $ Date                           <fct> 01/02/2018, 01/05/2018, 01/08/2018, 01/~
## $ Source                         <fct> AQS, AQS, AQS, AQS, AQS, AQS, AQS, AQS,~
## $ Site.ID                        <int> 370110002, 370110002, 370110002, 370110~
## $ POC                            <int> 1, 1, 1, 1, 1, 1, 1, 1, 1, 1, 1, 1, 1, ~
## $ Daily.Mean.PM2.5.Concentration <dbl> 2.9, 3.7, 5.3, 0.8, 2.5, 4.5, 1.8, 2.5,~
## $ UNITS                          <fct> ug/m3 LC, ug/m3 LC, ug/m3 LC, ug/m3 LC,~
## $ DAILY_AQI_VALUE                <int> 12, 15, 22, 3, 10, 19, 8, 10, 18, 7, 24~
## $ Site.Name                      <fct> Linville Falls, Linville Falls, Linvill~
## $ DAILY_OBS_COUNT                <int> 1, 1, 1, 1, 1, 1, 1, 1, 1, 1, 1, 1, 1, ~
## $ PERCENT_COMPLETE               <dbl> 100, 100, 100, 100, 100, 100, 100, 100,~
## $ AQS_PARAMETER_CODE             <int> 88502, 88502, 88502, 88502, 88502, 8850~
## $ AQS_PARAMETER_DESC             <fct> Acceptable PM2.5 AQI & Speciation Mass,~
## $ CBSA_CODE                      <int> NA, NA, NA, NA, NA, NA, NA, NA, NA, NA,~
## $ CBSA_NAME                      <fct> "", "", "", "", "", "", "", "", "", "",~
## $ STATE_CODE                     <int> 37, 37, 37, 37, 37, 37, 37, 37, 37, 37,~
## $ STATE                          <fct> North Carolina, North Carolina, North C~
## $ COUNTY_CODE                    <int> 11, 11, 11, 11, 11, 11, 11, 11, 11, 11,~
## $ COUNTY                         <fct> Avery, Avery, Avery, Avery, Avery, Aver~
## $ SITE_LATITUDE                  <dbl> 35.97235, 35.97235, 35.97235, 35.97235,~
## $ SITE_LONGITUDE                 <dbl> -81.93307, -81.93307, -81.93307, -81.93~
\end{verbatim}

\begin{Shaded}
\begin{Highlighting}[]
\FunctionTok{glimpse}\NormalTok{(EPAO319)}
\end{Highlighting}
\end{Shaded}

\begin{verbatim}
## Rows: 10,592
## Columns: 20
## $ Date                                 <fct> 01/01/2019, 01/02/2019, 01/03/201~
## $ Source                               <fct> AirNow, AirNow, AirNow, AirNow, A~
## $ Site.ID                              <int> 370030005, 370030005, 370030005, ~
## $ POC                                  <int> 1, 1, 1, 1, 1, 1, 1, 1, 1, 1, 1, ~
## $ Daily.Max.8.hour.Ozone.Concentration <dbl> 0.029, 0.018, 0.016, 0.022, 0.037~
## $ UNITS                                <fct> ppm, ppm, ppm, ppm, ppm, ppm, ppm~
## $ DAILY_AQI_VALUE                      <int> 27, 17, 15, 20, 34, 34, 27, 35, 3~
## $ Site.Name                            <fct> Taylorsville Liledoun, Taylorsvil~
## $ DAILY_OBS_COUNT                      <int> 24, 24, 24, 24, 24, 24, 24, 24, 2~
## $ PERCENT_COMPLETE                     <dbl> 100, 100, 100, 100, 100, 100, 100~
## $ AQS_PARAMETER_CODE                   <int> 44201, 44201, 44201, 44201, 44201~
## $ AQS_PARAMETER_DESC                   <fct> Ozone, Ozone, Ozone, Ozone, Ozone~
## $ CBSA_CODE                            <int> 25860, 25860, 25860, 25860, 25860~
## $ CBSA_NAME                            <fct> "Hickory-Lenoir-Morganton, NC", "~
## $ STATE_CODE                           <int> 37, 37, 37, 37, 37, 37, 37, 37, 3~
## $ STATE                                <fct> North Carolina, North Carolina, N~
## $ COUNTY_CODE                          <int> 3, 3, 3, 3, 3, 3, 3, 3, 3, 3, 3, ~
## $ COUNTY                               <fct> Alexander, Alexander, Alexander, ~
## $ SITE_LATITUDE                        <dbl> 35.9138, 35.9138, 35.9138, 35.913~
## $ SITE_LONGITUDE                       <dbl> -81.191, -81.191, -81.191, -81.19~
\end{verbatim}

\begin{Shaded}
\begin{Highlighting}[]
\FunctionTok{glimpse}\NormalTok{(EPAO318)}
\end{Highlighting}
\end{Shaded}

\begin{verbatim}
## Rows: 9,737
## Columns: 20
## $ Date                                 <fct> 03/01/2018, 03/02/2018, 03/03/201~
## $ Source                               <fct> AQS, AQS, AQS, AQS, AQS, AQS, AQS~
## $ Site.ID                              <int> 370030005, 370030005, 370030005, ~
## $ POC                                  <int> 1, 1, 1, 1, 1, 1, 1, 1, 1, 1, 1, ~
## $ Daily.Max.8.hour.Ozone.Concentration <dbl> 0.043, 0.046, 0.047, 0.049, 0.047~
## $ UNITS                                <fct> ppm, ppm, ppm, ppm, ppm, ppm, ppm~
## $ DAILY_AQI_VALUE                      <int> 40, 43, 44, 45, 44, 28, 33, 41, 4~
## $ Site.Name                            <fct> Taylorsville Liledoun, Taylorsvil~
## $ DAILY_OBS_COUNT                      <int> 17, 17, 17, 17, 17, 17, 17, 17, 1~
## $ PERCENT_COMPLETE                     <dbl> 100, 100, 100, 100, 100, 100, 100~
## $ AQS_PARAMETER_CODE                   <int> 44201, 44201, 44201, 44201, 44201~
## $ AQS_PARAMETER_DESC                   <fct> Ozone, Ozone, Ozone, Ozone, Ozone~
## $ CBSA_CODE                            <int> 25860, 25860, 25860, 25860, 25860~
## $ CBSA_NAME                            <fct> "Hickory-Lenoir-Morganton, NC", "~
## $ STATE_CODE                           <int> 37, 37, 37, 37, 37, 37, 37, 37, 3~
## $ STATE                                <fct> North Carolina, North Carolina, N~
## $ COUNTY_CODE                          <int> 3, 3, 3, 3, 3, 3, 3, 3, 3, 3, 3, ~
## $ COUNTY                               <fct> Alexander, Alexander, Alexander, ~
## $ SITE_LATITUDE                        <dbl> 35.9138, 35.9138, 35.9138, 35.913~
## $ SITE_LONGITUDE                       <dbl> -81.191, -81.191, -81.191, -81.19~
\end{verbatim}

\hypertarget{wrangle-individual-datasets-to-create-processed-files.}{%
\subsection{Wrangle individual datasets to create processed
files.}\label{wrangle-individual-datasets-to-create-processed-files.}}

\begin{enumerate}
\def\labelenumi{\arabic{enumi}.}
\setcounter{enumi}{2}
\item
  Change date columns to be date objects.
\item
  Select the following columns: Date, DAILY\_AQI\_VALUE, Site.Name,
  AQS\_PARAMETER\_DESC, COUNTY, SITE\_LATITUDE, SITE\_LONGITUDE
\item
  For the PM2.5 datasets, fill all cells in AQS\_PARAMETER\_DESC with
  ``PM2.5'' (all cells in this column should be identical).
\item
  Save all four processed datasets in the Processed folder. Use the same
  file names as the raw files but replace ``raw'' with ``processed''.
\end{enumerate}

\begin{Shaded}
\begin{Highlighting}[]
\CommentTok{\# 3}
\NormalTok{EPAPM19}\SpecialCharTok{$}\NormalTok{Date }\OtherTok{\textless{}{-}} \FunctionTok{mdy}\NormalTok{(EPAPM19}\SpecialCharTok{$}\NormalTok{Date)}
\NormalTok{EPAPM18}\SpecialCharTok{$}\NormalTok{Date }\OtherTok{\textless{}{-}} \FunctionTok{mdy}\NormalTok{(EPAPM18}\SpecialCharTok{$}\NormalTok{Date)}
\NormalTok{EPAO319}\SpecialCharTok{$}\NormalTok{Date }\OtherTok{\textless{}{-}} \FunctionTok{mdy}\NormalTok{(EPAO319}\SpecialCharTok{$}\NormalTok{Date)}
\NormalTok{EPAO318}\SpecialCharTok{$}\NormalTok{Date }\OtherTok{\textless{}{-}} \FunctionTok{mdy}\NormalTok{(EPAO318}\SpecialCharTok{$}\NormalTok{Date)}
\CommentTok{\# 4}
\NormalTok{EPAPM19\_select }\OtherTok{\textless{}{-}} \FunctionTok{select}\NormalTok{(EPAPM19, Date, DAILY\_AQI\_VALUE, Site.Name, AQS\_PARAMETER\_DESC,}
\NormalTok{    COUNTY, SITE\_LATITUDE, SITE\_LONGITUDE)}
\NormalTok{EPAPM18\_select }\OtherTok{\textless{}{-}} \FunctionTok{select}\NormalTok{(EPAPM18, Date, DAILY\_AQI\_VALUE, Site.Name, AQS\_PARAMETER\_DESC,}
\NormalTok{    COUNTY, SITE\_LATITUDE, SITE\_LONGITUDE)}
\NormalTok{EPAO319\_select }\OtherTok{\textless{}{-}} \FunctionTok{select}\NormalTok{(EPAO319, Date, DAILY\_AQI\_VALUE, Site.Name, AQS\_PARAMETER\_DESC,}
\NormalTok{    COUNTY, SITE\_LATITUDE, SITE\_LONGITUDE)}
\NormalTok{EPAO318\_select }\OtherTok{\textless{}{-}} \FunctionTok{select}\NormalTok{(EPAO318, Date, DAILY\_AQI\_VALUE, Site.Name, AQS\_PARAMETER\_DESC,}
\NormalTok{    COUNTY, SITE\_LATITUDE, SITE\_LONGITUDE)}
\CommentTok{\# 5}
\NormalTok{EPAPM19\_select}\SpecialCharTok{$}\NormalTok{AQS\_PARAMETER\_DESC }\OtherTok{\textless{}{-}} \StringTok{"PM2.5"}
\NormalTok{EPAPM18\_select}\SpecialCharTok{$}\NormalTok{AQS\_PARAMETER\_DESC }\OtherTok{\textless{}{-}} \StringTok{"PM2.5"}
\CommentTok{\# 6}
\FunctionTok{write.csv}\NormalTok{(EPAPM19\_select, }\StringTok{"./Data/Processed/EPAPM19\_Processed.csv"}\NormalTok{)}
\FunctionTok{write.csv}\NormalTok{(EPAPM18\_select, }\StringTok{"./Data/Processed/EPAPM18\_Processed.csv"}\NormalTok{)}
\FunctionTok{write.csv}\NormalTok{(EPAO318\_select, }\StringTok{"./Data/Processed/EPAO318\_Processed.csv"}\NormalTok{)}
\FunctionTok{write.csv}\NormalTok{(EPAO319\_select, }\StringTok{"./Data/Processed/EPAO319\_Processed.csv"}\NormalTok{)}
\end{Highlighting}
\end{Shaded}

\hypertarget{combine-datasets}{%
\subsection{Combine datasets}\label{combine-datasets}}

\begin{enumerate}
\def\labelenumi{\arabic{enumi}.}
\setcounter{enumi}{6}
\item
  Combine the four datasets with \texttt{rbind}. Make sure your column
  names are identical prior to running this code.
\item
  Wrangle your new dataset with a pipe function (\%\textgreater\%) so
  that it fills the following conditions:
\end{enumerate}

\begin{itemize}
\item
  Include all sites that the four data frames have in common: ``Linville
  Falls'', ``Durham Armory'', ``Leggett'', ``Hattie Avenue'', ``Clemmons
  Middle'', ``Mendenhall School'', ``Frying Pan Mountain'', ``West
  Johnston Co.'', ``Garinger High School'', ``Castle Hayne'', ``Pitt
  Agri. Center'', ``Bryson City'', ``Millbrook School'' (the function
  \texttt{intersect} can figure out common factor levels - but it will
  include sites with missing site information\ldots)
\item
  Some sites have multiple measurements per day. Use the
  split-apply-combine strategy to generate daily means: group by date,
  site name, AQS parameter, and county. Take the mean of the AQI value,
  latitude, and longitude.
\item
  Add columns for ``Month'' and ``Year'' by parsing your ``Date'' column
  (hint: \texttt{lubridate} package)
\item
  Hint: the dimensions of this dataset should be 14,752 x 9.
\end{itemize}

\begin{enumerate}
\def\labelenumi{\arabic{enumi}.}
\setcounter{enumi}{8}
\item
  Spread your datasets such that AQI values for ozone and PM2.5 are in
  separate columns. Each location on a specific date should now occupy
  only one row.
\item
  Call up the dimensions of your new tidy dataset.
\item
  Save your processed dataset with the following file name:
  ``EPAair\_O3\_PM25\_NC1819\_Processed.csv''
\end{enumerate}

\begin{Shaded}
\begin{Highlighting}[]
\CommentTok{\# 7}
\NormalTok{binded\_select }\OtherTok{\textless{}{-}} \FunctionTok{rbind}\NormalTok{(EPAO318\_select, EPAO319\_select, EPAPM18\_select, EPAPM19\_select)}

\CommentTok{\# 8}
\NormalTok{binded\_select1 }\OtherTok{\textless{}{-}}\NormalTok{ binded\_select }\SpecialCharTok{\%\textgreater{}\%}
    \FunctionTok{filter}\NormalTok{((binded\_select}\SpecialCharTok{$}\NormalTok{Site.Name }\SpecialCharTok{==} \StringTok{"Linville Falls"}\NormalTok{) }\SpecialCharTok{|}\NormalTok{ (binded\_select}\SpecialCharTok{$}\NormalTok{Site.Name }\SpecialCharTok{==}
        \StringTok{"Durham Armory"}\NormalTok{) }\SpecialCharTok{|}\NormalTok{ (binded\_select}\SpecialCharTok{$}\NormalTok{Site.Name }\SpecialCharTok{==} \StringTok{"Leggett"}\NormalTok{) }\SpecialCharTok{|}\NormalTok{ (binded\_select}\SpecialCharTok{$}\NormalTok{Site.Name }\SpecialCharTok{==}
        \StringTok{"Hattie Avenue"}\NormalTok{) }\SpecialCharTok{|}\NormalTok{ (binded\_select}\SpecialCharTok{$}\NormalTok{Site.Name }\SpecialCharTok{==} \StringTok{"Clemmons Middle"}\NormalTok{) }\SpecialCharTok{|}\NormalTok{ (binded\_select}\SpecialCharTok{$}\NormalTok{Site.Name }\SpecialCharTok{==}
        \StringTok{"Mendenhall School"}\NormalTok{) }\SpecialCharTok{|}\NormalTok{ (binded\_select}\SpecialCharTok{$}\NormalTok{Site.Name }\SpecialCharTok{==} \StringTok{"Frying Pan Mountain"}\NormalTok{) }\SpecialCharTok{|}
\NormalTok{        (binded\_select}\SpecialCharTok{$}\NormalTok{Site.Name }\SpecialCharTok{==} \StringTok{"West Johnston Co."}\NormalTok{) }\SpecialCharTok{|}\NormalTok{ (binded\_select}\SpecialCharTok{$}\NormalTok{Site.Name }\SpecialCharTok{==}
        \StringTok{"Garinger High School"}\NormalTok{) }\SpecialCharTok{|}\NormalTok{ (binded\_select}\SpecialCharTok{$}\NormalTok{Site.Name }\SpecialCharTok{==} \StringTok{"Castle Hayne"}\NormalTok{) }\SpecialCharTok{|}\NormalTok{ (binded\_select}\SpecialCharTok{$}\NormalTok{Site.Name }\SpecialCharTok{==}
        \StringTok{"Pitt Agri. Center"}\NormalTok{) }\SpecialCharTok{|}\NormalTok{ (binded\_select}\SpecialCharTok{$}\NormalTok{Site.Name }\SpecialCharTok{==} \StringTok{"Bryson City"}\NormalTok{) }\SpecialCharTok{|}\NormalTok{ (binded\_select}\SpecialCharTok{$}\NormalTok{Site.Name }\SpecialCharTok{==}
        \StringTok{"Millbrook School"}\NormalTok{)) }\SpecialCharTok{\%\textgreater{}\%}
    \FunctionTok{group\_by}\NormalTok{(Date, Site.Name, AQS\_PARAMETER\_DESC, COUNTY) }\SpecialCharTok{\%\textgreater{}\%}
    \FunctionTok{mutate}\NormalTok{(}\AttributeTok{mean\_AQI =} \FunctionTok{mean}\NormalTok{(binded\_select}\SpecialCharTok{$}\NormalTok{DAILY\_AQI\_VALUE), }\AttributeTok{mean\_latitude =} \FunctionTok{mean}\NormalTok{(binded\_select}\SpecialCharTok{$}\NormalTok{SITE\_LATITUDE),}
        \AttributeTok{mean\_longitude =} \FunctionTok{mean}\NormalTok{(binded\_select}\SpecialCharTok{$}\NormalTok{SITE\_LONGITUDE), }\AttributeTok{Month =} \FunctionTok{month}\NormalTok{(Date),}
        \AttributeTok{Year =} \FunctionTok{year}\NormalTok{(Date)) }\SpecialCharTok{\%\textgreater{}\%}
    \FunctionTok{select}\NormalTok{(}\SpecialCharTok{{-}}\FunctionTok{c}\NormalTok{(SITE\_LATITUDE, SITE\_LONGITUDE, DAILY\_AQI\_VALUE))}

\CommentTok{\# 9}

\NormalTok{binded\_select1\_spread }\OtherTok{\textless{}{-}} \FunctionTok{pivot\_wider}\NormalTok{(binded\_select1, }\AttributeTok{names\_from =}\NormalTok{ AQS\_PARAMETER\_DESC,}
    \AttributeTok{values\_from =}\NormalTok{ mean\_AQI)}
\end{Highlighting}
\end{Shaded}

\begin{verbatim}
## Warning: Values from `mean_AQI` are not uniquely identified; output will contain
## list-cols.
## * Use `values_fn = list` to suppress this warning.
## * Use `values_fn = {summary_fun}` to summarise duplicates.
## * Use the following dplyr code to identify duplicates.
##   {data} %>%
##   dplyr::group_by(Date, Site.Name, COUNTY, mean_latitude, mean_longitude,
##   Month, Year, AQS_PARAMETER_DESC) %>%
##   dplyr::summarise(n = dplyr::n(), .groups = "drop") %>%
##   dplyr::filter(n > 1L)
\end{verbatim}

\begin{Shaded}
\begin{Highlighting}[]
\CommentTok{\# binded\_select1\_spread$Ozone \textless{}{-} as.numeric(binded\_select1\_spread$Ozone)}

\CommentTok{\# 10}

\FunctionTok{dim}\NormalTok{(binded\_select1\_spread)}
\end{Highlighting}
\end{Shaded}

\begin{verbatim}
## [1] 8976    9
\end{verbatim}

\begin{Shaded}
\begin{Highlighting}[]
\CommentTok{\# 11}
\FunctionTok{write.csv}\NormalTok{(binded\_select1, }\AttributeTok{file =} \StringTok{"./Data/Processed/EPAair\_O3\_PM25\_NC1819\_Processed.csv"}\NormalTok{)}
\end{Highlighting}
\end{Shaded}

\hypertarget{generate-summary-tables}{%
\subsection{Generate summary tables}\label{generate-summary-tables}}

\begin{enumerate}
\def\labelenumi{\arabic{enumi}.}
\setcounter{enumi}{11}
\item
  Use the split-apply-combine strategy to generate a summary data frame.
  Data should be grouped by site, month, and year. Generate the mean AQI
  values for ozone and PM2.5 for each group. Then, add a pipe to remove
  instances where mean \textbf{ozone} values are not available (use the
  function \texttt{drop\_na} in your pipe). It's ok to have missing mean
  PM2.5 values in this result.
\item
  Call up the dimensions of the summary dataset.
\end{enumerate}

\begin{Shaded}
\begin{Highlighting}[]
\CommentTok{\# 12}
\NormalTok{binded\_select1\_summary }\OtherTok{\textless{}{-}}\NormalTok{ binded\_select1\_spread }\SpecialCharTok{\%\textgreater{}\%}
    \FunctionTok{group\_by}\NormalTok{(Site.Name, Month, Year) }\SpecialCharTok{\%\textgreater{}\%}
    \FunctionTok{mutate}\NormalTok{(}\AttributeTok{mean\_AQI\_ozone =} \FunctionTok{mean}\NormalTok{(binded\_select1\_spread}\SpecialCharTok{$}\NormalTok{Ozone), }\AttributeTok{mean\_AQI\_PM =} \FunctionTok{mean}\NormalTok{(binded\_select1\_spread}\SpecialCharTok{$}\NormalTok{PM2}\FloatTok{.5}\NormalTok{)) }\SpecialCharTok{\%\textgreater{}\%}
    \FunctionTok{drop\_na}\NormalTok{(mean\_AQI\_ozone)}
\end{Highlighting}
\end{Shaded}

\begin{verbatim}
## Warning: There were 616 warnings in `mutate()`.
## The first warning was:
## i In argument: `mean_AQI_ozone = mean(binded_select1_spread$Ozone)`.
## i In group 1: `Site.Name = Bryson City`, `Month = 1`, `Year = 2018`.
## Caused by warning in `mean.default()`:
## ! argument is not numeric or logical: returning NA
## i Run ]8;;ide:run:dplyr::last_dplyr_warnings()dplyr::last_dplyr_warnings()]8;; to see the 615 remaining warnings.
\end{verbatim}

\begin{Shaded}
\begin{Highlighting}[]
\CommentTok{\# 13}
\FunctionTok{dim}\NormalTok{(binded\_select1\_summary)}
\end{Highlighting}
\end{Shaded}

\begin{verbatim}
## [1]  0 11
\end{verbatim}

\begin{enumerate}
\def\labelenumi{\arabic{enumi}.}
\setcounter{enumi}{13}
\tightlist
\item
  Why did we use the function \texttt{drop\_na} rather than
  \texttt{na.omit}?
\end{enumerate}

\begin{quote}
Answer: drop\_na will let you remove rows where the column value is na
na.omit will not look at specefic value and delete all the values in all
rows and columns where the value is na
\end{quote}

\end{document}
